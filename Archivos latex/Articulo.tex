%--------------------------------------------------------------
%     Ejemplo de un art�culo con LaTeX
%--------------------------------------------------------------
\documentclass[a4paper]{article}
\usepackage[latin1]{inputenc}
\usepackage[spanish]{babel}
\usepackage{amssymb}
\usepackage{amsthm}
\usepackage{amsmath}
%--------------------------------------------------------------
\title{Ejemplo de un documento cient�fico con \LaTeX\, realizado con el estilo \texttt{article}}
\author{Nombre Autor 1
\thanks{Agradece que hayan elegido \LaTeX\ para editar los textos.}
\and Nombre Autor 2\thanks{Espera que este ejemplo sea de
utilidad.}}
\date{Fecha de creaci�n}
%--------------------------------------------------------------
\begin{document}
\maketitle
\begin{abstract}
Se trata de un simple ejemplo de como se estructura un art�culo y de
como se puede realizar un �ndice de contenidos de las diferentes
partes del mismo.
\end{abstract}
\tableofcontents
%--------------------------------------------------------------
\section{Aspectos b�sicos de \LaTeX}

\subsection{Esquema de funcionamiento}

\subsubsection{Instalaci�n de programas}

\subsection{Fichero fuente}
\subsubsection{Pre�mbulo}
\subsubsection{Cuerpo}
\dots


\section{Composici�n de un documento}


\section{Tipos y colores}


\subsection{Unos tipos}


\subsection{Otros tipos}


\subsection{Unos colores}


\subsection{Otros colores}


\section*{Ahora esto no lo numero, luego no aparecer� en el �ndice}

\section{Ahora vuelvo a numerar}

\section{Fin de la parte principal}


\appendix
\section{Primera parte del ap�ndice}

\section{Segunda parte del ap�ndice}

\section{�ltima parte del ap�ndice}


\end{document}
