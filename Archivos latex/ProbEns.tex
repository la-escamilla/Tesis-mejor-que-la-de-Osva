%--------------------------------------------------------------
%--------------------------------------------------------------
\documentclass[a4paper]{article}
\usepackage[latin1]{inputenc}
\usepackage[spanish]{babel}
\usepackage{amssymb}
\usepackage{amsthm}
\usepackage{amsmath}
%--------------------------------------------------------------
\title{Problem�ticas de la ense�anza de la historia de las ciencias b�sicas exactas y naturales}

\date{}
%--------------------------------------------------------------
\begin{document}
\maketitle
\begin{abstract}
Los estudiantes de las Ciencias B�sicas Exactas y Naturales (C.B.E y N.) se encuentran en un problema de estructuraci�n de conceptos cient�ficos debido a la ineficacia en los cursos elementales, as� como el desinter�s por parte de los docentes por complementar sus c�tedras con la debida historia que engloba cada concepto.
En especial, se visualiza una proyecci�n lineal de la historia documentada y expuesta y cuando los estudiantes intentan visualizar dichas ideas, caen en el aburrimiento y desinter�s, pues necesitan hacer un an�lisis profundo de los hechos expuestos. En contraparte, los profesionistas de las C.B.E y N. carecen de elementos clave para organizar, planear y realizar investigaciones documentales rigurosas.
El problema, dando enfoque al �rea de las matem�ticas, engloba la falta de divulgaci�n de las fascinantes investigaciones que se dan en esta ciencia y la imagen social que generan las clases de profesionistas que no se consolidan integralmente como docentes, los cuales, repercuten en esta misma. Esta imagen es la idea de que las matem�ticas son aburridas.
\end{abstract}

%--------------------------------------------------------------
\textbf{Palabras/Frases clave}
\\Historia de la ciencia, desinter�s, aburrimiento, falta de divulgaci�n.\\
\textbf{Referencias.}
\\.[1]Didactica della Facolta di Scinze Matematiche,Fisiche e Naturali. Universita di Pavia
\\.[2]Finley, F.,Allchin,D.,Rhees,D.,Fifield,S. (eds.).(1995).Proceeding of the Third International History, Philosophy and Science Teaching Conference. Minneapolis, MN: University of Minnesota.
\\.[3]Gil P�rez, D.(1993). Contribuci�n de la Historia y de la Filosof�a de las Ciencias al Desarrollo de un Modelo de Ense�anza/Aprendizaje como Investigaci�n. Ense�anza de las Ciencias, 11(2),197-212
\\.[4]S�nchez Ron, J. M. (1988). Usos y Abusos de la Historia de la F�sica en la Ense�anza.
\\.[5]SEP (1993). Educaci�n b�sica. Secundaria. Plan y programas de estudio. Secretar�a de Educaci�n P�blica, M�xico, D.F.
\\.[6]Slisko,J.(1997a). Experimentos de Galileo  en Libros de Texto de F�sica para Secundaria. trabajo presentado en II Convenci�n Nacional de Profesores de Ciencias Naturales, Ixtapan de la Sal, Marzo de 1997. 
\\.[7]Slisko, J.(1997b). La Corona de Her�n en la Ense�anza de la F�sica, Bolet�n de la Sociedad Mesicana de F�sica, 11(4), 231-232.
\end{document}
