%--------------------------------------------------------------
%--------------------------------------------------------------
\documentclass[a4paper]{article}
\usepackage[latin1]{inputenc}
\usepackage[spanish]{babel}
\usepackage{amssymb}
\usepackage{amsthm}
\usepackage{amsmath}
%--------------------------------------------------------------
\title{Pit�goras en la biolog�a}

\date{}
%--------------------------------------------------------------
\begin{document}
\maketitle
\begin{abstract}
as biol�gicos(en este caso �rboles) para optimizar sus recursos y lograr de esta manera obtener mejores resultados.
\end{abstract}

%--------------------------------------------------------------
\textbf{Palabras/Frases clave}
\\Teorema de Pit�goras, di�metro, bifurcaci�n, �rboles.\\
\textbf{Referencias.}
\\.[1]universofractal.blogspot.mx/2012/12/arbol-de-pitagoras.html
\\.[2]Enquist, B. J., Universal scaling in tree and vascular plant allometry: toward a general quantitative theory linking plant and function from cells to ecosystems, Tree Physiology 11, 1044-1064(2002)
\\.[3]McMahon, T., Size and Shape in Biology, Science, 179, 1201-1204(1973)
\\.[4]West G. B., Brown J. H., Life's Universal Scaling Laws. Physics Today, (2004)
\end{document}
