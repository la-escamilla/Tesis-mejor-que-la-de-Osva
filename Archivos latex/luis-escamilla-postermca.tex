%--------------------------------------------------------------
%--------------------------------------------------------------
\documentclass[a4paper]{article}
\usepackage[latin1]{inputenc}
\usepackage{amssymb}
\usepackage{amsthm}
\usepackage{amsmath}
%--------------------------------------------------------------
\title{Pit�goras en la biolog�a}
\date{}
%--------------------------------------------------------------
\begin{document}
\maketitle
\begin{abstract}
Los agujeros negros son regiones en el espacio-tiempo que generan un campo gravitatorio tal que ninguna part�cula material, ni siquiera la luz, puede escapar. Son uno de los objetos astrof�sicos m�s misteriosos debido a que no pueden ser observados a simple vista y no se ha determinado con certeza el proceso para su formaci�n ni su composici�n. Sin embargo, con base a las observaciones se ha podido determinar algunas propiedades fundamentales como: la masa, ubicaci�n y recientemente (11 de febrero de 2016) la colaboraci�n LIGO detect� por primera vez ondas gravitacionales debido al choque de dos agujeros negros. Aunque la formaci�n de agujeros negros se relaciona habitualmente con el colapso gravitacional de una estrella lo suficientemente masiva, en los primeros instantes del Big Bang, de acuerdo con el modelo est�ndar de part�culas, hubo regiones donde la densidad era tan grande que pudieron haberse formado agujeros negros primordiales debido al colapso gravitacional. Esta idea fue publicada por primera vez en 1966 por Zel?Dovich y Novik. En este trabajo, de acuerdo con la premisa de la existencia de dichos agujeros negros primordiales, se har� un breve repaso a su historia, las teor�as y modelos que soportan estos fen�menos y c�mo se han propuesto como candidatos a la materia oscura. 
\end{abstract}

%--------------------------------------------------------------
\textbf{Palabras/Frases clave}
\\Teorema de Pit�goras, di�metro, bifurcaci�n, �rboles.\\
\textbf{Referencias.}
\\.[1]universofractal.blogspot.mx/2012/12/arbol-de-pitagoras.html
\\.[2]Enquist, B. J., Universal scaling in tree and vascular plant allometry: toward a general quantitative theory linking plant and function from cells to ecosystems, Tree Physiology 11, 1044-1064(2002)
\\.[3]McMahon, T., Size and Shape in Biology, Science, 179, 1201-1204(1973)
\\.[4]West G. B., Brown J. H., Life's Universal Scaling Laws. Physics Today, (2004)
\end{document}
